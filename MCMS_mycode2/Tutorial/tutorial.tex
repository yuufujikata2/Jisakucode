%\documentclass{slides}
\documentclass[landscape]{slides}
\usepackage{graphics}
\usepackage{color}
%\usepackage{psfig}
\usepackage{epsfig}
\newcommand{\drawline}{\vspace{-.5em}
\begin{center}
--------------------------------------------------------
\end{center}
\vspace{-.5em}
}

\pagestyle{empty}
\addtolength{\textheight}{4cm}
\addtolength{\topmargin}{-2.5cm}
\addtolength{\textwidth}{1cm}
\addtolength{\oddsidemargin}{-.5cm}

\newcommand{\vs}{{\phantom{*}}}
\newcommand{\page}{\end{slide}\begin{slide}}

\begin{document}

\begin{slide}
{\bf The Multichannel Multiple Scattering (MCMS) Program} 

\hfill [Peter Kr\"uger, 2016]

- currently only for $L_{2,3}$-edge XAS with particle--hole wave function

- magnetic systems possible but not shown here

Calculated spectra [{\tt spectrum.dat}]:

- isotropic spectrum  \quad $I_{\rm iso}(\omega) = \sum_q I_q(\omega)$

- polarization: linear x, y, z, circular left/right (+/-) for ${\bf k}\sim z$

- arbitrary polarisation can be found from 
 $I_{qq'}(\omega)$  [{\tt specqqprime.dat}]\\
e.g. circular dichroism for ${\bf k}\sim z$, $I_{CD} = I_{++} - I_{--}$


- spectral decomposition in final state $d$-orbitals 
$I_{mm'}(\omega)$ [{\tt specmmprime.dat}]


\page
{\bf Some features of the code}

\begin{itemize}
\item most parts written in ANSI C
-- dynamical memory allocation.

\item calls LAPACK library routines (in Fortran 77)

\item calls CONTINUUM [C. R. Natoli] for $G^{ij}_{LL'}$ \\
 -- written in Fortran 77 with static memory allocation
\end{itemize}

\newpage
{\bf Program Flow}

\begin{itemize}
\item {\tt readinput} \quad (FILE mcms.in)

\item {\tt readlmtopot} \quad (absorber potential FILEs)

\item {\tt readcofi} \quad (core wave function FILE)

\item construct partially screened potential

\item if the FILE {\tt rhochi.dat} exists, read it

({\tt rhochi.dat} contains reflectivity)

\item otherwise call {\tt environment}
which calculates reflectivity on (fine) energy mesh \{  $\epsilon_i$
\} through standard M.S. calculation
\begin{itemize}

\item read {\tt instr.dat} = structure file 

\item {\tt readlmtopot} \quad (all potential FILEs)

For all photoelectron energies $\epsilon_i$:

\item  calculate $t$-matrices and radial matrix elements
of all atoms

\item  call {\tt CONTINUUM} 
which calculates $G^{ij}_{LL'}$ and sets up MS matrix.

\item invert MS matrix

\item return reflectivity($\epsilon_i$) to {\tt main}

\end{itemize}


\item
calculate basis orbitals (open + closed) for R-matrix

\item
make electronic configurations and N-electron basis states =
Slater-determinants -- no angular momentum coupling

\item
calculate matrix elements of hamiltonian between Slater determinants\\
-- most complicated are those of Coulomb operator
% integrals $(\nu 2p|r^k/r^{k+1}|\nu' 2p)$  and angular matrix elements 


\item
calculate (N-electron) matrix elements of dipole operator

\item
calculate (N-electron) matrix elements of Q and L


{\it Begin total (or photon) energy E loop}

\item
Solve eigen-channel eqs (=generalized eigenvalue problem)
\[
\sum_{\nu} (E-H-L)_{\mu\nu} c_{\nu k}  = \sum_{\nu} Q_{\mu\nu} c_{\nu k} b_k 
\]
\item $\rightarrow$ R-matrix

\item match with Bessel, Neumann-functions\\
$\rightarrow$ 
${\cal T}^{-1}$ = inverse of multichannel T-matrix (E)

\item
$M = {\cal T}^{-1} -\rho$\quad $\rightarrow$ pick up $\rho(\epsilon_\alpha)$ by interpolating $\rho(\epsilon)$

\item invert $M$ \quad $\Rightarrow$ \quad $\tau^{00}_{\alpha\beta}$

\item
calculate dipole matrix elements $M_{\alpha L}=\langle \Psi_{\alpha L}|r_q|\Psi_g \rangle$ 

\item
calculate XAS cross section \quad 
$I_q(E) = -\sum_{\alpha\beta} M_\alpha^*\; {\rm Im} \tau^{00}_{\alpha\beta}\; M_\beta$  

{\it End total (or photon) energy E loop}
\end{itemize}








\page
{\bf Input files}

$\bullet$ 1 structure file: \quad {\tt instr.dat}

% $\bullet$ 1 potential file for each type of atom: {\tt Ca O \dots}

$\bullet$ 1 potential file for each inequivalent atom in the system:\\
e.g. CaO: at least \ {\tt Ca, O}\\
better 1 extra file for final-state rule potential (fully relaxed)
{\tt Ca\_ch}

$\bullet$ core-orbital file

$\bullet$ main input file \quad {\tt mcms.in}\\
contains parameters of calculation


\page
{\bf The structure file}  \qquad {\tt  instr.dat}

\begin{small}
\begin{verbatim}
   27   27    2
 Ca_01  3  20    0.000000    0.000000    0.000000
  O_02  1   8    0.000000   -4.544792    0.000000
 Ca_03  2  20    0.000000    4.544792    4.544792
  O_04  1   8   -4.544792    4.544792    4.544792
  O_02  1   8   -4.544792    0.000000    0.000000
  O_02  1   8    0.000000    0.000000    4.544792
  O_02  1   8    4.544792    0.000000    0.000000
  O_02  1   8    0.000000    0.000000   -4.544792
  O_02  1   8    0.000000    4.544792    0.000000
 Ca_03  2  20    4.544792    0.000000   -4.544792
  ...
  O_04  1   8    4.544792   -4.544792   -4.544792
  ...
\end{verbatim}
\end{small}

1st line:
{\tt number of atoms, number of atoms, 2}

all other lines (as many as atoms):
\begin{verbatim}
 potfilename    lmax    Z    x    y    z
\end{verbatim}

(x,y,z in Bohr radii = 0.529177 Angs)

\page
{\bf The potential files}  \qquad {\tt  any\_name}

{\bf EITHER}:\\
use LMTO potential files

\begin{small}
\begin{verbatim}
GENERAL: LMX=2  NSPIN=1   WSR= 2.300000  REL=F  NR=309  A=.030
   Z=22  QC=18  QTOT=-1.657339
...
POT:
  309    1     0.03000     2.30000
 1.446896229E+02 1.446896215E+02 1.446896172E+02 1.446896098E+02 1.446899148E+02
 ...
 ...
 1.948503960E+01 1.893767293E+01 1.839966378E+01 1.787094852E+01
...
\end{verbatim}
\end{small}

\end{slide}
\begin{slide}

{\bf OR}:\\ generate ``LMTO-formatted'' files from {\tt POTGEN/PHAGEN}
potential files using utility program {\tt readhspot}.\\
-- Take a potential file (in potgen format): e.g. xalpha.pot\\
-- Run the programs

{\tt transformhspot < xalpha.pot > C\_readable.pot}

{\tt readhspot C\_readable.pot}

This generates the input file {\tt instr.dat} and one potential-file for
each inequivalent atom in LMTO format. These files are named 
{\tt X\_01}  {\tt Y\_02} etc where X,Y are the symbols of the elements.



\begin{small}
\begin{verbatim}
LMTO-type pot-file generated from H.S. pot-file
Z= 20
POT:
  321    1     0.03000     2.99000
 0.000000000e+00 9.726327137e+01 9.726687378e+01 9.727058367e+01 9.727440416e+01
 ...
 1.420510254e+01 1.377548513e+01 1.335201620e+01 1.293418401e+01 1.252140073e+01
 1.211308769e+01
\end{verbatim}
\end{small}

\end{slide}
\begin{slide}
{\bf The main input file} \qquad {\tt  mcms.in}

$\bullet$ format:

{\tt KEYWORD VALUE}  [any comment] {\tt KEYWORD VALUE} \dots

$\bullet$ order irrelevant

$\bullet$ some variables have default values  (\dots danger \dots)

$\bullet$ comments at any place
{\bf except}
 between a {\tt KEYWORD} and its {\tt VALUE}

$\bullet$ energy unit: {\bf Rydberg} on input,
{\bf eV} on output (spectra)

$\bullet$ {\tt KEYWORD}'s are {\bf case sensitive} 

\end{slide}
\begin{slide}
{\it Example of} \quad {\tt mcms.in}
\begin{verbatim}
AbsCHPotFile     Ca_ch     % final state potential file
CoreWFFile       ca_2p_ch  % core wave function file    
PhotoelEMin     -0.61      % photo electron energy range and mesh
PhotoelEMax      1.29
DeltaPhotoelE    0.005
KsiCor           0.25      % spin-orbit parameter = 2 Delta / 3
ReVmtz          -0.61604   % zero of muffin-tin potential
OmegaMin         26.3      % photon energy range and mesh
OmegaMax         27.5
DeltaOmega       0.005
UnscreenedWeight 0.        % mixing parameter for screening
SlaterRkScale    0.        % possible scaling of Coulomb integrals
BasisEmin       -1.        % range for searching for trial orbitals
BasisEmax        1.
WithRefl         1         % if = 0  atomic calculation (rho = 0)
\end{verbatim}

\end{slide}

\begin{slide}
{\bf Output files}

$\bullet$ A lot of information is written on screen.\\ 
It seems useful to redirect the screen output to some file.\\
{\tt ./mcms > output}

$\bullet$ if {\tt rhochi.dat} exists, it is taken as input.\\
If not, it is output. Contains reflectivity data.

$\bullet$ {\tt spectrum.dat} is the main spectrum output file.\\
The meaning of the columns is given in the first line:
\begin{verbatim}
# omega [eV], Iiso=Ix+Iy+Iz, Ix, Iy, Iz, I-, I+, omega+ecore [eV]
\end{verbatim}
{\tt omega} = photon energy, {\tt omega+ecore} = photoelectron energy.

\end{slide}
\begin{slide}
$\bullet$ {\tt specqqprime.dat} \quad contains the full tensor of
polarisation dependent cross section, i.e. $I_{qq'}$
with $q=0,\pm 1$.\\
First column: photoelectron energy = omega+ecore[eV].\\
$I_{qq'}$ is in column number $3q+q'+6$. 
 

$\bullet$ {\tt specmmprime.dat} \quad contains $I_{mm'}$
the cross section projected on final state d-orbitals $m$ 
(diagonal and off-diagonal terms)
\begin{center}
\begin{tabular}{r|ccccc}
m   & -2 & -1 & 0 & 1 & 2 \\
orbital & $ixy$ & $iyz$ & $3z^2$-$r^2$ & $zx$ & $x^2$-$y^2$\\
\end{tabular}
\end{center}
First column: photoelectron energy = omega+ecore[eV].\\
$I_{mm'}$ is in column number $5m+m'+14$. 


\end{slide}
\begin{slide}

\begin{center}
{\large\bf Tutorial -- L23-edge spectra of CaO}
\end{center}
Already prepared (directory {\tt Input}) 

-- potential files  (from an LMTO band structure
calculation):\\ {\tt Ca}, {\tt O}, {\tt Ca\_ch} 
(= fully screened core-hole potential).

-- core wave function file \quad {\tt ca\_2p\_ch}

-- input file \quad {\tt mcms.in}


{\bf CaO6 octahedron}

$\bullet$ write an {\tt instr.dat} file for a CaO6 octahedron

Choose {\tt lmax = 3} for Ca.

CaO has rocksalt structure.\quad
(FCC with Ca at (0,0,0), O at (.5,.5,.5)).

Lattice constant a = 4.81 \AA\ = 9.09 a.u.


$\bullet$ check your cluster.
You can run the program {\tt instr2xyz}
which reads {\tt instr.dat} and generates {\tt str.xyz}
(=structure file in XYZ format and \AA).
The latter can be used 
in most visualization software.

\begin{small}
{\em If you fail writing your own {\tt instr.dat} file, look up {\tt MoreInput}}
\end{small}

$\bullet$ run the code to calculate a spectrum in the independent
particle aprroximation ({\tt SlaterRkScale   0.})

{\tt ./mcms > output}

$\bullet$ have a look at the {\tt output} file

$\bullet$ plot the isotropic XAS spectrum from {\tt spectrum.dat} 

$\bullet$ check for linear dichroism by comparing $Ix$, $Iy$, $Iz$\\
e.g. $Iz-Ix$ in gnuplot: {\tt plot 'spectrum.dat' using 1:(\$5-\$3)}

$\bullet$ analyze and assign the peaks in terms of d-orbital symmetries
by plotting various columns of {\tt specmmprime.dat}.
What do you expect from group theory?

{\bf Tetragonal distortion}

$\bullet$ remove or rename {\tt rhochi.dat} 

$\bullet$ perform a tetragonal distortion of the structure, e.g.\ 
by some $+4$\% in z and $-2$\% in x and y (bond lengths 4.635 and 4.365 a.u.)\\
Repeat all the analyses and explain the observations.


{\it Go back to the undistorted CaO6 cluster}

$\bullet$ compare the isotropic spectrum with experiment


{\bf Core-hole effect -- final state rule}

$\bullet$ try to improve the spectrum by using 
different core-hole potentials

-- {\tt AbsCHPotFile  Ca} and {\tt UnscreenedWeight 0.}\\
-- {\tt AbsCHPotFile  Ca\_ch} and {\tt UnscreenedWeight 0.}\\
-- {\tt AbsCHPotFile  Ca\_ch} and {\tt UnscreenedWeight 0.1}


$\bullet$ use {\tt AbsCHPotFile  Ca\_ch} and {\tt UnscreenedWeight 0.1}\\
and switch on the multiplet coupling ({\tt SlaterRk 1.0})

$\bullet$ plot the isotropic spectrum and compare with experiment

$\bullet$ analyze and assign the peaks in terms of d-orbitals
by plotting various columns of {\tt specmmprime.dat}

{\bf 27-atom cluster}

$\bullet$  rename {\tt instr.dat\_27} to {\tt instr.dat} 

$\bullet$ visualize this cluster

$\bullet$ calculate isotropic spectrum and compare with CaO6 cluster

\begin{small}
{\it Be  careful to choose sufficiently fine energy meshes (both for photon $\omega$ and photoelectron $\epsilon$) because 
the peaks get sharper with increasing cluster size (why?)}
\end{small}

%{\bf EXTRA: TiCl4}
 
%Look up the bond length of TiCl4 and generate the instr.dat file.
%Try to generate potentials using the potgen\_scf code by C. R.Natoli, 
%which is used in various other codes taught in this school, 
%so you should be familiar with it. 
%Study the influence of the electron-hole coupling.


% {\small\it Whenever you change the structure, don't forget to rename or remove {\tt rhochi.dat}}
\end{slide}
\end{document}
